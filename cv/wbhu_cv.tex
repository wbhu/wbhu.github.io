%----------------------------------------------------------------------------------------
%	PACKAGES AND OTHER DOCUMENT CONFIGURATIONS
%----------------------------------------------------------------------------------------

\documentclass{resume} % Use the custom resume.cls style

\usepackage[left=0.75in,top=0.6in,right=0.75in,bottom=0.6in]{geometry} % Document margins
\usepackage[colorlinks,linkcolor=blue]{hyperref}
\newcommand{\tab}[1]{\hspace{.2667\textwidth}\rlap{#1}}
\newcommand{\itab}[1]{\hspace{0em}\rlap{#1}}
\name{Wenbo HU} % Your name
\address{Rm 1026, SHB, CUHK, Hong Kong} % Your address
\address{(+86)15012537485 \\ \href{mailto:wbhu@cse.cuhk.edu.hk}{wbhu@cse.cuhk.edu.hk} \\ GitHub: \href{https://github.com/wbhu}{github.com/wbhu}} % Your phone number and email

\begin{document}

%----------------------------------------------------------------------------------------
%	EDUCATION SECTION
%----------------------------------------------------------------------------------------

\begin{rSection}{Education}

{\bf The Chinese University of Hong Kong} \hfill {\em August 2018 - Present} 
\\ Department of Computer Science and Engineering
\\ Ph.D. Student\hfill{Supervised by \href{https://appsrv.cse.cuhk.edu.hk/~ttwong/myself.html}{Tien-Tsin Wong}}
\\

{\bf Dalian University of Technology} \hfill {\em September 2014 - June 2018}
\\ Department of Computer Science and Technology
\\ Bachelor of Engineering.\hfill { Rank: 3/105, GPA:90.1/100 }

\end{rSection}

\begin{rSection}{RESEARCH INTERESTS}
My general research interests are computer graphics, computer vision and deep learning. Currently, I am specially interested in 3D deep learning, including stereo image/video editing and 3D scene understanding. 
\end{rSection}

%----------------------------------------------------------------------------------------
%	WORK EXPERIENCE SECTION
%----------------------------------------------------------------------------------------

\begin{rSection}{Experience}

\begin{rSubsection}{SenseTime, Beijing, China}{Jan 2018 - Jul 2018}{Research Intern}{Supervised by \href{https://wuwei-ai.org/}{Wu Wei}}
\item Develop model to enhance low-light image/video to help to boost high-level vision tasks.
\item Develop model to handle face verification under backlit environment.
\item Develop model to remove reflections on glasses.
\end{rSubsection}

\begin{rSubsection}{DJI, Shenzhen, China}{Jul 2016 - Sep 2016}{Summer Intern}{}
\item Based on DJI M100, design software to make it automatically takeoff, landing, hovering, going to specific position in the room, grasping dolls and dropping them into a narrow box in room environment.
\item Develop algorithm to fuse the data from IMU, Visual Odometry and ultrasonic sensors.
\end{rSubsection}

\end{rSection}
%--------------------------------------------------------------------------------
%    Projects And Seminars
%-----------------------------------------------------------------------------------------------
\begin{rSection}{Current Projects}

{\bf Joint 2D-3D semantic segmentation}
\\Although sparse convolution has greatly reduced the memory usage while applying deep CNN on 3D data for semantic segmentation, 3D segmentation still suffer from the low quality of 3D data, i.e., SOTA method on ScanNet benchmark can only get 28.6 IoU for "picture" class currently, since the quality of picture region in reconstructed point clouds is very low. On the other hand, images have much higher quality while 2D semantic segmentation is lack of geometry information. We plan to jointly do 2D and 3D semantic segmentation by projecting and back projecting information among 2D and 3D space and expect it can boost both 2D and 3D results.\\

\end{rSection}


\begin{rSection}{Publications}

{\bf [3] Deep Line Drawing Vectorization via Line Subdivision and Topology Reconstruction}
\\Y. Guo, Z. Zhang, C. Han, \textbf{W. Hu}, C. Li and T. T. Wong\\
Computer Graphics Forum (special issue on PG 2019), October 2019\\ 

{\bf [2] DEMC: A Deep Dual-Encoder Network for Denoising Monte Carlo Rendering}
\\X. Yang, \textbf{W. Hu}, D. Wang, L. Zhao, B. Yin, Q. Zhang, X. Wei, H. Fu\\
Computational Visual Media Conference (CVM), 2019\\ 

{\bf [1] Fast Reconstruction for Monte Carlo Rendering Using Deep Convolutional Networks}
\\X. Yang, D. Wang,\textbf{W. Hu}, L. Zhao, X. Piao, D. Zhou, Q. Zhang, B. Yin, Q. Cai, X. Wei\\
IEEE Access, The Multidisciplinary Open Access Journa, 2018\\

\end{rSection}

%----------------------------------------------------------------------------------------
%	TECHNICAL STRENGTHS SECTION
%----------------------------------------------------------------------------------------

\begin{rSection}{PROFESSIONAL ACTIVITIES}
\item \textbf{Conference Reviewer:}
	\subitem MMM2020 (sub-reviewer)

\end{rSection}


\begin{rSection}{Skills}
\item \textbf{Languages:} Python, MATLAB, C/C++
\item \textbf{Toolkits:} PyTorch, TensorFlow, Git
\end{rSection}


\begin{rSection}{Teaching}

\begin{tabular}{ @{} >{\bfseries}l @{\hspace{6ex}} l }
CSCI1120: Introduction to Computing Using C++ \ & Spring, 2019 - 2020 \\
CSCI3280: Introduction to Multimedia Systems \ & Fall, 2018 - 2019 \\
\end{tabular}

\end{rSection}


\end{document}
